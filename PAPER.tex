\documentclass[11pt]{iopart}
\usepackage{graphicx}
\usepackage{subfigure}
\usepackage{morefloats}
\usepackage{color}

\begin{document}

\title{\textbf{Profile measurements in the plasma edge of MAST using a ball pen probe}}
\author{ N R Walkden$^{1,2}$, J Adamek$^{3}$, S Allen$^{1}$, B D Dudson$^{2}$, S Elmore$^{1}$, \\G Fishpool$^{1}$, A Kirk$^{1}$ and M Komm $^{3}$
        \\ \small{$^{1}$ EURATOM/CCFE Fusion Association, Culham Science Centre, Abingdon, OX14 3DB, UK} 
        \\ \small{$^{2}$ York Plasma Institute, Department of Physics, University of York, Heslington, York, YO10 5DD, UK} 
        \\ \small{$^{3}$ CPP...}
        \\ Email: \texttt{nrw504@york.ac.uk} }
\date{}

\begin{abstract}

\end{abstract}

\section{Introduction}
The plasma potential plays an important role in many of the processes occuring in the plasma edge. In the drift ordering of the two-fluid equations appropriate to the plasma edge \cite{SimakovPoP2003,SimakovPoP2004} the $\textbf{E}\times\textbf{B}$ velocity provides the dominant advective flow and determines the dynamics of both turbulence \cite{MilitelloPPCF2013} and individual filaments \cite{WalkdenPPCF2013}. Despite its important role in edge physics the plasma potential has remained a difficult quantity to measure accurately on the fluctuation timescale. The ball pen probe (BPP), developed by Adamek \emph{et.al} \cite{AdamekCJP2004,AdamekCJP2005} offers a diagnostic technique capable of measuring the plasma potential with a robust implementation that may withstand plasma fluxes up to and within the seperatrix. The BPP technique has been tested against an emissive probe \cite{AdamekCJP2005} and a self-emitting langmuir probe \cite{AdamekPreprint} and shows excellent agreement in each case. The robustness of the BPP design against the high particle and heat fluxes in the plasma edge, as well as its relative simplicity to implement makes it an attractive option over the emissive probe, which can rarely be operated near to the seperatrix due to its inherently weak structural design. The BPP technique has been implemented on the CASTOR tokamak \cite{AdamekCJP2004,AdamekCJP2005,SchritweisserCJP2006}, on ASDEX-Upgrade \cite{AdamekJNM2009,AdamekCPP2010,HoracekNF2010} and in the low temperature plasma device TJ-K \cite{AdamekCPP2013}. 
\\In this paper results are presented from the development and implementation of a BPP on the Mega Amp Spherical Tokamak (MAST). MAST, being a tight aspect ratio tokamak, has a reduced magnetic field strength compared to both ASDEX-Upgrade and CASTOR but maintains electron and ion temperatures in the same region as the former two. As a consequence the ion (and electron) Larmor radius, $\rho_{i}$ on MAST is larger than in the previous two devices, but the ions remain fully magnetized, unlike TJ-K. This makes MAST an ideal intermediate test of the BPP technique. Figure 1 shows $\rho_{i}$ calculated for MAST compared to figures quoted for ASDEX-Upgrade and CASTOR.
\begin{figure}
\caption{}
\end{figure}
In section 2 the BPP design for MAST is described. In section three plasma potential profile measurements are presented and compared to the standard floating potential measurement from a Langmuir probe. In section 4 the electron temperature calculated from the BPP signals is presented and a comparison with the Thomson scattering diagnostic on MAST is used to determine the success of the BPP measurement. Furthermore in section 4 a simple heuristic model for BPP collection is described with a view towards producing non-empirically based correction factors for the BPP measurements. Finally in section 5 the BPP measurement is used to derive the radial electric field. Section 6 provides a discussion and a summary. 

\section{BPP implementation on MAST}  



\section{References}
\bibliographystyle{prsty}
\bibliography{Paper-ref}


\end{document}
