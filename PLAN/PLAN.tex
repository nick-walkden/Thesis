\documentclass[11pt]{article}






\begin{document}

\title{Thesis Plan}
\author{Nick Walkden}
\date{}
\maketitle


\section{Titles}

\section{Chapters}

\subsection{Introduction}
\subsection{Experimental Review}
\subsection{Theoretical Review}
\newpage
\subsection{Ball Pen Probe}
This chapter will be a description of the design, modelling and experimental results from the ball pen probe. It will include the following work:
\begin{enumerate}
\item A description of the ball pen probe implementation on MAST \\ \\
\begin{tabular}{l l}
\hline
\textbf{Completion status:} &  Ready for writeup \\
\hline
  \textbf{Time required:} & N/A \\
\hline
\textbf{Tasks:} & Write up section
\end{tabular}
\item Present modelling work towards understanding the BPP collection mechanism \\ \\
\begin{tabular}{l l}
\hline
\textbf{Completion status:} &  Finished analytics, need to solve numerically \\
\hline
  \textbf{Time required:} & 2 Weeks \\
\hline
\textbf{Tasks:} & Write integrator for $I^{+}/I^{-}$ \\
 & Complete analysis of probe impedance \\
 & Attempt to compare probe impedance to experiment
\end{tabular}
\item Potential Profile measurements\\ \\
\begin{tabular}{l l}
\hline
\textbf{Completion status:} &  Ready for writeup \\
\hline
  \textbf{Time required:} & N/A \\
\hline
\textbf{Tasks:} & Write up section
\end{tabular}
\item Temperature profile measurements and assesment of BPP technique for measurement\\ \\
\begin{tabular}{l l}
\hline
\textbf{Completion status:} & Temperature profiles available \\
 & Need to conduct assesment of $\alpha$\\
\hline
  \textbf{Time required:} & 2 Weeks \\
\hline
\textbf{Tasks:} & Refine Thomson comparison \\
 & Critically asses $\alpha$ 
\end{tabular}
\item Radial electric field measurements\\ \\
\begin{tabular}{l l}
\hline
\textbf{Completion status:} &  Figures available, awaiting comparison data \\
\hline
  \textbf{Time required:} & 2 Weeks \\
\hline
\textbf{Tasks:} & Compare BPP E-field to DBS and ECELESTE
\end{tabular}
\item Fluctuation measurements in shear layer region\\ \\
\begin{tabular}{l l}
\hline
\textbf{Completion status:} &  Not yet started \\
\hline
  \textbf{Time required:} & 3 Weeks \\
\hline
\textbf{Tasks:} & Write PDF and moment calculator\\
 & Perform analysis of PDF in shear layer
\end{tabular}



\end{enumerate}
\newpage
\subsection{Two-Dimensional Blob Modelling}
This chapter will provide a detailed introduction to 2D blob modelling followed by a description of the work carried out on the basic physics underlying 2D blob theory including the following:
\begin{enumerate}
\item Introduction to 2D blob simulations in intertially limited, sheath limited and coherent motion regime\\ \\
\begin{tabular}{l l}
\hline
\textbf{Completion status:} &  Simulations ready \\
 & Plots need refining \\
\hline
\textbf{Time required:} & $< 1$ Week \\
\hline
\textbf{Tasks:}  & Generate figures \\
 & Write up section
\end{tabular}
\item Analytic comparison of inertially limited blobs/holes velocity including plasma background \\ \\
\begin{tabular}{l l}
\hline
\textbf{Completion status:} &  Analytics complete, Simulations ready \\
\hline
\textbf{Time required:} & 2 Weeks \\
\hline
\textbf{Tasks:}  & Refine comparison plots \\
 & Test blob/hole length scale velocity scaling
\end{tabular}
\item Effect of blob anisotropy on propagation dynamics\\ \\
\begin{tabular}{l l}
\hline
\textbf{Completion status:} &  Inital simulations carried out \\
\hline
\textbf{Time required:} & 2 Weeks \\
\hline
\textbf{Tasks:}  & Test a range of anisotropy parameter \\
 & Attempt comparison with analytic theory
\end{tabular}
\item Blob - Shear layer interaction\\ \\
\begin{tabular}{l l}
\hline
\textbf{Completion status:} &  Code ready \\
 & Need to develop length-scale based analytic work \\
 & Need to prepare shear layer profile from expt\\
 & Need to carry out simulations\\
\hline
\textbf{Time required:} & 4 Weeks \\
\hline
\textbf{Tasks:}  & Figure out analytic treatment \\
 & Develop input from expt \\
 & Carry out simulations 
\end{tabular}
\end{enumerate}
\newpage
\subsection{Three-Dimensional MAST Modelling}
This chapter will be a description of 3D filament using Angus's work as a starting point. It will include the following work:
\begin{enumerate}
\item 3D filament simulations in a slab with the basic model as an introduction and to recreate Angus's work in MAST. \\ \\
\begin{tabular}{l l}
\hline
\textbf{Completion status:} &  Started \\
\hline
  \textbf{Time required:} & 2 Weeks \\
\hline
\textbf{Tasks:} &
      Set up slab simulation with MAST parameters
\\ &  Run over a range of $\Delta$ and $\delta$
\end{tabular}
\item 3D filament simulations in a slab with hot-ions \\ \\
\begin{tabular}{l l}
\hline
\textbf{Completion status:} &  Initial work implementing hot ions complete \\
 & Need to carry out simulations \\
\hline
  \textbf{Time required:} & 4 Weeks \\
\hline
\textbf{Tasks:} &
      Go through Angus drift wave work including effects of hot ions
\\ & Complete symmetry breaking analysis \\
 &  determine parameters for simulations
\\ &  Run simulations
\end{tabular}
\item 3D filament simulations in a slab with parallel streaming \\ \\
\begin{tabular}{l l}
\hline
\textbf{Completion status:} &  Parallel streaming implemented and under testing \\
\hline
  \textbf{Time required:} & 4 Weeks \\
\hline
\textbf{Tasks:} &
      Apply linear analysis to system 
\\ &  Apply symmetry analysis to system\\
 & Finish testing and carry out simulations
\end{tabular}
\item 3D flux tube simulations with the basic model\\ \\
\begin{tabular}{l l}
\hline
\textbf{Completion status:} & Nearly Completed \\
\hline
  \textbf{Time required:} & $\sim$ 3 days \\
\hline
\textbf{Tasks:} & Re-run  with shadowing\\
  &     Write up section
\end{tabular}
\item 3D flux tube simulatations with the full isothermal model\\ \\
\begin{tabular}{l l}
\hline
\textbf{Completion status:} & Not yet started \\
\hline
  \textbf{Time required:} & 4 Weeks \\
\hline
\textbf{Tasks:} & Finish stability checking of hot-ion parallel streaming model\\
 & Decide on correct starting conditions\\ 
 & Carry out simulations
\end{tabular}
\item 3D slab simulations with divertor tilt boundary conditions\\ \\
\begin{tabular}{l l}
\hline
\textbf{Completion status:} &  Not yet started \\
\hline
  \textbf{Time required:} & 4 Weeks \\
\hline
\textbf{Tasks:} & Implement Loizu boundary conditions in BOUT++\\
 & Carry out simulations varying $\delta$
\end{tabular}
\item 3D full SOL simulations with the full isothermal model and divertor plate tilt\\ \\
\begin{tabular}{l l}
\hline
\textbf{Completion status:} &  Started on grid generator \\
\hline
  \textbf{Time required:} & 4 Weeks \\
\hline
\textbf{Tasks:} & Continue to develop FWA grid generator\\
 & Carry out simulations in a realistic scenario
\end{tabular}
\end{enumerate}
\newpage
\subsection{Three-Dimensional MAST-Upgrade Modelling}
This chapter will extend the results of the previous chapter to the case of the MAST-U Super-X diverter geometry. In particular it will address the effect of an extended connection length on filament dynamics. It will include:
\begin{enumerate}
\item An investigation in 2D of the effect of increased connection length on the transition from inertially limited to sheath limited blob dynamics\\ \\
\begin{tabular}{l l}
\hline
\textbf{Completion status:} &  Code ready \\
 & Need to set up and run simulation scans \\
\hline
  \textbf{Time required:} & 1 Week \\
\hline
\textbf{Tasks:} & Run Simulations \\
 & Review sheath-limited theory for analytic comparison
\end{tabular}
\item 3D SXD flux tube simulations over a parameter scan\\ \\
\begin{tabular}{l l}
\hline
\textbf{Completion status:} & Grid needs developing  \\
 & Code ready \\
 & Simulations need carrying out\\
\hline
  \textbf{Time required:} & 4 Weeks \\
\hline
\textbf{Tasks:} & Generate grid with suitable resolution in divertor region \\
 & Run flux-tube simulation for different psi in SXD geometry
\end{tabular}
\item Full SOL SXD simulations\\ \\
\begin{tabular}{l l}
\hline
\textbf{Completion status:} & Grid generator started  \\
 & Simulations need carrying out\\
\hline
  \textbf{Time required:} & 4 Weeks \\
\hline
\textbf{Tasks:} &  Continue to develop FWA grid generator\\
 & Carry out simulations in a realistic scenario
\end{tabular}
\end{enumerate}
\subsection{Zero-Dimensional Predictive Modelling}
\subsection{Conclusions}
\subsection{References}
\subsection{Appendix: The BOUT++ Code}








\end{document}
